\section{Background}

\subsection{Biology}
\label{sec:biology}

DNA, deoxyribonucleic acid, is a polymeric molecule that consists of the four
chemically distinct nucleotides adenine (A), cytosine (C), guanine (G) and
thymine (T). RNA, ribonucleic acid, is also a polymeric molecule where the
nucleotide thymine is replaced by uracil (U).  The nucleotides can be linked
together in any order to form chains that are up to millions of nucleotides
long~\cite[pp.~8--9]{brown}. These chains are called sequences.

Every organism has a genome that contains the biological information that is
needed to construct and maintain a living example of that organism. For humans
and most other cellular life forms, the genome is made up of DNA, but a few
viruses have RNA genomes~\cite[pp.~3--4]{brown}.

Genomes are changing over time as a result of continuous small-scale sequence
alterations. These are called \textit{mutations}. Mutations happen because of
DNA replication errors or from damaging effects of mutagens, such as chemicals
or radiation. The mutations can be \textit{point mutations} which are single
nucleotide edits. Others include deletions of a section in the DNA and
insertions, where one or more nucleotides are inserted into a new place in the
DNA~\cite[pp.~505--506]{brown}. Another type of mutation is when a segment is
moved from one part of the sequence to another. These segments are known as
\emph{transposable elements}~\cite{munoz}.

Clustering can serve to identify DNA or RNA sequences that has been mutated and
group mutated sequences of the same origin into the same cluster with just a
single representative sequence.
