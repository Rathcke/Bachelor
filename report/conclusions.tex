\section{Project summary}
\label{sec:conclusions}

We have presented an efficient, greedy, heuristic clustering algorithm, along
with an efficient algorithm using $k$-mer counting as an approximation for
calculating sequence similarity. The presented clustering algorithm and
distance metric have been implemented in the program \texttt{klust}, which is
open-source and freely available, cf. section \ref{sec:introduction}.

The program \texttt{klust} has shown to give results of high quality, a low
number of resulting clusters and can perform as well or better than
\texttt{USEARCH} in terms of both speed and memory usage. However, the
clustering algorithm is almost exhaustive in the search for a centroid, since
the $max\_rejects$ parameter is not effectively used. Although the centroid
search is almost exhaustive, actual sequence comparisons are almost only
performed when they will succeed anyway and there are barely any false
negatives, i.e. skipped comparisons which would have yielded a match.

The implementation of the presented distance metric is very fast and works
well with mutations such as chunks of changes and transpositions, although it
could be less sensitive to larger chunks of changes and single nucleotide
edits.


\subsection{Future work}
\label{sec:future_work}

While the implementation is efficient and produces good clustering results,
some key features could be improved.

The most important improvement is the search for good centroid matches. As of
now, the implementation does not use the parameter $max\_rejects$, or $m$, in
an effective manner. It is very rare that a sequence reaches the maximum number
of rejects, and as such it will go through the entire centroid list if it does
not have a match in the centroid list. Instead of just ordering the centroids
based on the intersection of $k$-mers as \texttt{USEARCH} does, an idea is to
expand on the implementation of links between centroids. Instead of having only
one link, it could have $m$ links or at least be able to trace through the
links to $m$ centroids that make good potential matches. The idea is to build a
graph that connects centroids that are close to each other.  While this might
sound heavy, it should not add much to the running time as the graph can be
constructed while clustering. Each iteration would take a bit longer, but when
clustering huge amounts of data, the sacrifice would quickly pay off. This way,
the sequences that would become singletons, are quickly dismissed. As it uses
most of the time searching for centroids, this is one of the most important
things to optimize. However, due to time constraints, we have not been able to
do much research on this, so it is hard to say if it will produce a lot more
clusters or if it will be slower. It could, however, improve the scaling that
happens when the threshold increases.

Another improvement is related to the distance metric. As of now it is only
insensitive to a few types of mutations. It still penalizes large chunk edits
(insertions or deletions) a lot more than it should. It should be able to
detect these mutations and not punish them as much as it does. While locating
these segments can be computationally hard, calculating the distance is very
cheap in the implementation and relatively few comparisons are made compared to
the number of iterations over centroids. The additional computations to cover
insertion and deletion mutations could provide a more correct similarity
measure, in a biological sense, i.e. better identification of actual genomic
mutations.

With regard to the implementation, there are also some future improvements to
be made: The $k$-mer frequency vectors could be stored together with the
centroids and reused in subsequent similarity computations. Additionally, there
can be made more checks for the possibility of stopping a comparison early, if
it is already known that it will fail. Another interesting part of the
implementation is the parallelization of the clustering, which could be
developed further so that it would yield fewer clusters, e.g. using a better
merge strategy. In addition, the memory management could be improved so that
output would be written continuously and sequences deallocated as soon as they
become part of an existing cluster.
