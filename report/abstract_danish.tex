\begin{abstract}
  \noindent
  Denne afhandling præsenterer en effektiv, grådig og heuristisk algoritme til
  at producere clusters af store mængder DNA- og RNA-sekvenser sammen med en
  open source implementering af algoritmen.

  Algoritmen benytter $k$-mers til at vurdere om to sekvenser hører til samme
  cluster. For at bestemme om to sekvenser faktisk hører til samme cluster,
  bruges en similaritetsmål baseret på $k$-mers som giver en approksimation af
  lighed mellem sekvenser. Dette reducerer tidskompleksiteten til lineær i
  længden af den længste sekvens istedet for det mere almindeligt brugte
  similaritetsmål sequence alignment som har en kvadratisk tidskompleksitet.
  Det er dette mål der blandt andet bliver brugt i proprietære program
  \texttt{USEARCH}.

  Det implementerede program, \texttt{klust}, er blevet testet på syntetisk
  sekvens data samt data fra den virkelige verden, som består af mere end $3$
  millioner sekvenser med en gennemsnitslængde på over \num{1000} nukleotider.

  Program opfører sig som forventet på det syntetiske data og producerer
  cluster resultater af høj kvalitet på ægte data.

  Programmet har vist sig at være hurtigere og producere færre clusters end \texttt{USEARCH} samtidig med at have et lavt hukommelsesforbrug
\end{abstract}