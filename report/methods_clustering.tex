\subsection{Cluster analysis algorithms}

There are several approaches to clustering sequence data. This section
describes different clustering paradigms. It discusses the strengths,
weaknesses and potentials of these methods.

\subsubsection{Hierarchical clustering}

There are typically two types of hierarchical clustering algorithms, the
agglomerative type that works in a bottom-up manner and the divisive type that
has a top-down approach. This section describes the agglomerative version. In
the agglomerative version all sequences start as leaves in a tree, where the
leaves are considered clusters. Each cluster is then merged together with the
closest cluster, thus building a tree, level for level, until there is only one
cluster left. This gives a hierarchy of the clusters. Alternatively, when the
clustering is based on a similarity threshold, as in this project, the clusters
are merged until no more clusters can be merged. The algorithm is visualized in
Figure \ref{fig:hierarchical_clustering}.

\begin{figure}[h!]
  \centering
  \def\svgwidth{\columnwidth}
  \import{graphics/}{Hierarchical_Clustering.pdf_tex}
  \caption{Illustration of agglomerative hierarchical clustering. Clusters,
    with objects represented as letters, are merged with nearby clusters based
    on lexicographic order until there is only one cluster.}
  \label{fig:hierarchical_clustering}
\end{figure}

This addresses the question of what is required for a cluster to be ``close''
to another. Besides measuring similarity between sequences with some distance
metric, a definition for when clusters should be merged is needed. The
following describes different approaches for merging two clusters based on a
similarity threshold.

In the \textit{complete-linkage} approach, all sequences from one cluster must
have a similarity above the threshold to all sequences from the other. In the
\textit{average-linkage}, the average similarity between all pairs of
sequences is required to be above the threshold. With the \textit{single-
linkage} approach, only two sequences, one from each cluster, need an identity
above the similarity threshold.

The complete-linkage approach is computationally very heavy as each merge will
take $\mathcal{O}(mn)$ time, where $m$ and $n$ are the sizes of the clusters.
The average-linkage method is equally computationally hard. This is not viable
for a quick clustering algorithm.

However, the single-linkage approach is often used in sequence
clustering~\cite[pp. 62-63]{dong}. It has the advantage of behaving greedily
and stopping early if a link meets the similarity criterion instead of
calculating all links to find the minimal distance. Though the greedy behavior
can boost performance, it could make a very rough clustering as two sequences
from the clusters might be very similar while other sequences could have a very
low similarity to each other.

There exists an agglomerative method using single-linkage with complexity
$\mathcal{O}(n^2)$ known as SLINK~\cite{sibson} and another using
complete-linkage~\cite{defays}. As the naive agglomerative method, both
algorithms build a distance matrix and then uses row and column elimination but
is optimized to $\mathcal{O}(n^2)$ instead of $\mathcal{O}(n^3)$. Even though
$\mathcal{O}(n^2)$ is too heavy for the scope of this project, it produces
optimal results and builds the entire hierarchy. With a greedy single-linkage
approach and without building the entire hierarchy, the complexity could be
significantly reduced.  The distance matrix itself takes $n^2$ time to create,
which is already too much, so this indicates the computation of the distances
and therefore the merges should be an ongoing process with the clustering.


\subsubsection{Graph based clustering}

The general framework of graph-based clustering consists of two steps. First
step is to generate a weighted graph from the sequences. The second step, the
clustering step, is to split the graph into subgraphs, which correspond to the
clusters~\cite[pp. 64-65]{dong}.

Consider a graph $G$ with a set of vertices $V$ and a set of edges $E$. In
graph based clustering, we consider $V$ as the set of sequences and $E$ is the
set of similarity scores. The first step is to calculate all $E_{ij}$, the
similarity score between $V_i$ and $V_j$, based on some distance metric. If the
clustering is based on a similarity threshold, edges that are below the
similarity threshold can be ignored.

There are many variants to the clustering step. E. Hartuv and R.
Shamir~\cite{hartuv} presents an algorithm that tries to minimize edges that
connect two subgraphs by moving vertices from one subgraph to another.

H. Kawaji \textit{et al.}~\cite{kawaji} presented an algorithm that tries to
maximize the density of edges in a subgraph. This is done by converting the
tree into a simpler tree by removing all edges that are under the similarity
threshold.

Since both of these requires that all edges' weights have been computed, it is
already too slow for efficient sequence clustering. If the subgraphs could be
constructed while computing distances, the time complexity could be reduced.
Both hierarchical and graph based clustering indicates that a heuristic
clustering is needed to obtain the desired performance.


\subsubsection{Greedy algorithm like \texttt{UCLUST}}

The \texttt{UCLUST} algorithm is greedy and is designed so that all member
sequences have similarity greater than or equal to the similarity threshold,
to their centroid.  It works by processing one sequence at a time and
comparing these to the existing centroids. If a match is found, the sequence
is assigned to the cluster of the matching centroid, otherwise it becomes a
centroid of a new cluster.

It is designed to also support that each centroid has a similarity under the
threshold to all other centroids, but this is not guaranteed to hold. The
reason is that the algorithm only compares a sequence to a prespecified number
of centroids given by the $max\_reject$ parameter.

It uses the $k$-mer counts to locate centroid candidates for a match, though
how it does it is not very well described.

The similarity calculations are performed with global sequence alignments as R.
C. Edgar~\cite{usearch_algorithm}  claims that using the word counts to
calculate identity is not sensitive enough. Greedy clustering is illustrated in
Figure \ref{fig:greedy_clustering}.

\begin{figure}[H]
  \def\svgwidth{\columnwidth}
  \import{graphics/}{Greedy_Clustering.pdf_tex}
  \caption{Illustration of greedy clustering. Sequences are processed in the
    order they are read. They either become a centroid for a new cluster or
    they are assigned to an existing cluster.}
  \label{fig:greedy_clustering}
\end{figure}

Greedy clustering can reduce the time complexity as non-centroid sequences are
only handled once. This can also improve space complexity. The time complexity
is dependent on the number of centroids which is typically much lower than the
number of sequences. An efficient search for centroids can reduce the expected
running time to linear as opposed to the worst case $\mathcal{O}(n^2)$
complexity, where $n$ is the number of sequences (in the case where all
sequences become centroids).
