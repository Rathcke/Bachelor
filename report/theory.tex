\section{Theory}
\subsection{Starting out: Levenshtein with dynamic programming}
For a start, a naive version of the Levenshtein distance metric was
implemented, but as expected, this is absolutely useless in any practical
setting, since it does not reuse already calculated results. Therefore, this
was quickly transformed into a dynamic programming solution, which solves
subproblems just once, stores and reuses the intermediate results. This dynamic
programming bottom up solution was however still very slow, yielding a
performance of around 70 comparisons per second. This could most likely be
optimized further, but because of the characteristics of an edit distance
algorithm like Levenshtein and because of the performance requirements for this
project, we choose to not pursue further optimization and instead focus on
other types of algorithms.

\subsection{Simple $d2$ distance}
After learning that the basic Levenshtein algorithm is far too slow for the
problem domain of this project, we turned our attention to the $d2$ distance
metric. The first version of the $d2$ distance metric algorithm is shown in
figure \ref{alg:d2_naive}. This algorithm maintains a single frequency vector,
as a map structure to allow for large $k$ which results in a large number of
possible different $k$-mers. The map is indexed using the lexicographical
position of the $k$-mer. When iterating through the first sequence, $k$-mer
counts are incremented and when iterating through the second sequence, they are
decremented. Finally all the frequencies are squared and the square root of the
sum is returned, corresponding to the Euclidean distance between frequency
vectors for the two sequences.

\begin{algorithm}
  \caption{Naive \textsc{d2} distance metric}
  \label{alg:d2_naive}
  \begin{algorithmic}[1]
    \Require{$s$ and $t$ are DNA or RNA sequences and $k \in \mathbb{Z}^+$}
    \Statex
    \Function{d2}{$s, t, k$}
      \State initialize \texttt{freq\_map}
      \For{$i \gets 0$ to $length(s) - k$}
        \State freq\_map[index\_of\_kmer(s.substring(i, k)]\texttt{++}
      \EndFor
      \For{$i \gets 0$ to $length(t) - k$}
        \State freq\_map[index\_of\_kmer(t.substring(i, k)]\texttt{--}
      \EndFor
      \State $total \gets 0$
      \ForAll{$e \in$ freq\_map}
        \State $total \gets e.value^2$  \Comment{calculate the Euclidean distance}
      \EndFor
      \State \Return $\sqrt{total}$
    \EndFunction
  \end{algorithmic}
\end{algorithm}


\subsection{$d2$ distance with windows and Jaccard index}
The current version of the $d2$ distance function uses the concept of a
\emph{window}, of a certain length, that iterates over the sequences and
calculates distances between the substrings in each window. This calculation is
done using a kind of forward differences method for reducing the calculations
to a few fixed operations for calculating the distance in the next window from
the distance in the current window. This concept is described in
\cite{hazelhurst}.

The algorithm presented here uses a window size equal to the length of the
shortest of the two sequences. Let $|s|$ be the length of the shortest of the
two sequences. To begin with the, the $k$-mers in the first $|s|$ characters of
each sequence are counted and the \emph{Manhattan} distance between these two
frequency vectors is calculated.

The Manhattan distance is simply the Euclidean distance where squaring is
replaced with absolute value and the square root is omitted, i.e. for
$u, v \in \mathbb{Z}^n$,

\begin{equation}
  d_{Manhattan} \eqdef \sum_{i=1}^{n} |u_i - v_i| \;.
\end{equation}

This distance is the distance between the subsequences in the first position of
the window. To calculate the distance in the following window, i.e. advancing
the window through the longer of the two sequences by one character, it is
decided which $k$-mers exit and enter the window, respectively, and then by
looking at whether the existing $k$-mer count in the frequency vector is
negative or positive, it can be decided whether the distance increases or
decreased by 2 or whether it stays the same. Subsequently, the frequency vector
is updated to reflect the change in the new window. The following illustrates
the idea of a window and $k$-mers exiting and entering the window:

\begin{verbatim}
      |---- window ----------|
      ACTGATCGTAGCTAGCTAGTGTTG
      ACGTAGATCGTGGATGGCTGATCGTAGCTAAGCTTAGCTGATCG.....
      ^^^^                 ^^^^
      k-mer exiting        k-mer entering
\end{verbatim}

The Manhattan distance can be hard to use in practice, because it is very
dependent on the length of the shortest sequence (the length of the window),
the concept of the \emph{Jaccard index}, or the \emph{Jaccard similarity
coefficient}, is used to ``normalize'' the distance to a value in the interval
$[0,1]$.

The Jaccard index of two sets $A$ and $B$ is defined as follows:
\begin{equation}
  J(A, B) \eqdef \frac{|A \cup B| - |A \cap B|}{|A \cup B|}
\end{equation}
 % TODO:  maybe actually Jaccard distance; maybe add citation

In the context of $k$-mer frequencies, the union can be interpreted as the
total number of $k$-mers in the window in the two sequences, and the
intersection as the Manhattan distance in the window.

