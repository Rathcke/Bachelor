\section{Terminology}

The words \emph{sequence} and \emph{string} will be used to denote the same
concepts, i.e. an ordered list of objects, where an object will most often be a
text character. In general, a sequence or string might be infinite, but in this
project they will always be finite unless mentioned otherwise.

In this text, the notion of a subsequence is different from that of a
substring, as is the common convention: a \emph{substring} of a sequence $S$ is
a consecutive, ordered list of objects that occur in $S$, while a
\emph{subsequence} of $S$ is a sequence that can be obtained from $S$ by
deleting some objects from the sequence without changing the order of the
objects.

The \emph{distance} between sequences denotes some absolute measure of how far
the sequences are from each other in some well-defined space, e.g. which fixed
length substrings occur in the sequences and how many times. The
\emph{distance threshold} is the maximum distance between two sequences for
them to belong in the same cluster.

The \emph{similarity} or \emph{identity} between sequences denotes a relative
measure of the distance between the sequences, i.e. a normalized distance
measure where \num{0} means no similarity and \num{1} means perfect similarity. A
\emph{similarity threshold} or \emph{identity threshold} is a minimum
similarity that two sequences must have to belong in the same cluster.  When
two sequences are above a given similarity threshold, they are said to
\emph{match} one another.

The term \emph{query sequence} denotes a sequence for which a matching centroid
is sought and \emph{target sequence} denotes the centroid being compared to.

The name \texttt{USEARCH} denotes both the piece of software developed by
Robert C. Edgar and the algorithm used in the program for searching for
sequences in some database, e.g. for searching for a centroid that is likely
above a given distance threshold of some sequence. The name \texttt{UCLUST}
refers to the clustering algorithm used in the \texttt{USEARCH} program.


\subsection{Notation}

Let $s$ and $t$ be sequences.
\begin{itemize}
  \item $d(s,t)$ denotes the distance between $s$ and $t$
  \item $sim(s,t)$ denotes the similarity between $s$ and $t$
  \item $|s|$ denotes the length of $s$
  \item $K(s)$ denotes the set of $k$-mers in $s$
\end{itemize}
