\section{Introduction}

%Clustering is the task of grouping objects such that the objects in each group
%are more similar to each other than to objects in other groups and such that
%objects in different groups are not similar enough to be placed in the same
%group, according to some notion of similarity.

Clustering is the task of partitioning a set of objects, such that each
partition contains objects that are similar to each other and objects in
separate partitions are dissimilar, according to some well-defined notion of
similarity. These partitions are called \emph{clusters}.

This project is concerned with the clustering of DNA and RNA sequences. A
\emph{centroid} is a sequence that represents a cluster of sequences.

The clustering method described in this document strives to assign each
sequence to a cluster based on the similarity between the sequence and the
centroid. The clustering method does not guarantee a minimal number of
clusters, as the search for a centroid is not exhaustive, nor does it guarantee
to find the most similar centroid for each sequence.

Clustering huge amounts of DNA/RNA sequences (up to 500 million strings of
500—1500 characters each) is computationally hard and there are not many
available algorithms and tools for efficient clustering of sequencing data. The
one tool \texttt{UCLUST}\footnote{\url{http://drive5.com/usearch/}} that does
the job is closed-source and considered too expensive.

This project researches the possibilities for creating an open source tool that
can match the performance of the proprietary version of \texttt{UCLUST}. We
present an implementation of a solution that uses the concept of $k$-mer
counting, for measuring the distance between sequences, and utilizes the
$k$-mers occurring in the centroids for improving the search for a centroid that
is within the given threshold distance.
