\section{Introduction}

Clustering is the task of partitioning a set of objects, such that each
partition contains objects that are similar to each other and objects in
separate partitions are dissimilar, according to some well-defined notion of
similarity. These partitions are called \emph{clusters}.

Clustering huge amounts of DNA/RNA sequences (up to 500 million strings of
500—1500 characters each) is computationally hard, and the cost of sequencing,
i.e. determining DNA and RNA data, has dropped so low, and consequently the
amount of data has grown so large, that the available algorithms and hardware
can no longer keep up.~\cite{rothberg}

There are not many available algorithms and tools for efficient clustering of
sequencing data. The one tool which is currently doing the job,
\texttt{UCLUST}\footnote{\url{http://drive5.com/usearch/}}, is closed-source
and although the 32-bit version is available for free, the usable memory is
limited to 4 GB and the 64-bit version is costly.

As the amount of sequencing data grows, clusterings becomes increasingly more
relevant as it can be used to reduce a large set of sequences to a much smaller
set of representative sequences, called \emph{centroids}, which each represents
a single cluster.

This project is concerned with the clustering of DNA and RNA sequences. The
clustering method described in this document strives to assign each sequence to
a cluster based on the similarity between the sequence and the centroid. The
clustering method does not guarantee a minimal number of clusters, as the
search for a centroid is not exhaustive, nor does it guarantee to find the most
similar centroid for each sequence.

This project researches the possibilities for creating an open-source tool that
can match the performance of \texttt{UCLUST}. We present a solution and
implementation that uses the concept of $k$-mer counting, i.e. essentially
comparing the distribution of all possible substrings of a fixed length $k$ in
each sequence, as an approximation for measuring sequence similarity and which
utilizes the $k$-mers occurring in the centroids for improving, given a 
sequence, the search for a centroid that is likely to be a good representative for that sequence.

Earlier work have been very focused on techniques based on sequence alignment,
with \texttt{UCLUST} being one such example. However, a clustering algorithm
with a similarity measure based on $k$-mers, can possible give an improved
performance in terms of speed, while still being a sufficiently good approximation.

The program, named \texttt{klust}, has been implemented in \texttt{C++} and is
hosted on GitHub and available at \url{https://github.com/Rathcke/klust}.

The reader of this document is expected to have a basic knowledge of computer
science, corresponding to around two or three years of undergraduate studies,
however no knowledge of biology or bioinformatics is expected.
